\documentclass[conference]{IEEEtran}
\IEEEoverridecommandlockouts
% The preceding line is only needed to identify funding in the first footnote. If that is unneeded, please comment it out.
\usepackage{cite}
\usepackage{amsmath,amssymb,amsfonts}
\usepackage{algorithmic}
\usepackage{graphicx}
\usepackage{textcomp}
\usepackage{xcolor}
\usepackage{listings}
\usepackage{url}
\usepackage{float}
\usepackage{hyperref}
\hypersetup{
    colorlinks=true,
    urlcolor=blue,
    linkcolor=blue,
    citecolor=blue
}
\def\BibTeX{{\rm B\kern-.05em{\sc i\kern-.025em b}\kern-.08em
    T\kern-.1667em\lower.7ex\hbox{E}\kern-.125emX}}
\begin{document}

\title{Metric Learning: Triplet-Loss\\
}

\author{\IEEEauthorblockN{Khashayar Zardoui}
\IEEEauthorblockA{\textit{Dept. Computer Science \& Software Engineering} \\
\textit{Concordia University}\\
Montreal, Canada \\
khashayar.zardoui@mail.concordia.ca}
{\footnotesize ID: 40052568}
\and
\IEEEauthorblockN{Paolo Junior Angeloni}
\IEEEauthorblockA{\textit{Dept. Computer Science \& Software Engineering} \\
\textit{Concordia University}\\
Montreal, Canada \\
p\_ange@live.concordia.ca}
{\footnotesize ID: 25976944}
}

\maketitle


\section{The Triplet-Loss Pipeline}
In this supervised similarity or metric learning, the Triplet-Loss pipeline consists of 
\begin{enumerate}
    \item Retrieve images from CUB200\_2011 dataset within \texttt{TripletCUBDataset} class
    \item Transform images into tensors and apply additional augmentations to the training set only
    \item Train the model using the ResNet18 and ResNet34 pretrained models (20 epochs with varying learning rates and batch sizes)
    \item Optaining the results for the trainings
    \item Trainig the model with ResNet18 with 60 epochs (varying batch sizes)
    \item testing the models and optaing the test metrics and comparing the results
\end{enumerate}
\par\vspace{1em}