\documentclass[conference]{IEEEtran}
\IEEEoverridecommandlockouts
% The preceding line is only needed to identify funding in the first footnote. If that is unneeded, please comment it out.
\usepackage{cite}
\usepackage{amsmath,amssymb,amsfonts}
\usepackage{algorithmic}
\usepackage{graphicx}
\usepackage{textcomp}
\usepackage{xcolor}
\usepackage{listings}
\usepackage{url}
\usepackage{float}
\usepackage{hyperref}
\hypersetup{
    colorlinks=true,
    urlcolor=blue,
    linkcolor=blue,
    citecolor=blue
}
\def\BibTeX{{\rm B\kern-.05em{\sc i\kern-.025em b}\kern-.08em
    T\kern-.1667em\lower.7ex\hbox{E}\kern-.125emX}}
\begin{document}

\title{Metric Learning: Triplet-Loss\\
}

\author{\IEEEauthorblockN{Khashayar Zardoui}
\IEEEauthorblockA{\textit{Dept. Computer Science \& Software Engineering} \\
\textit{Concordia University}\\
Montreal, Canada \\
khashayar.zardoui@mail.concordia.ca}
{\footnotesize ID: 40052568}
\and
\IEEEauthorblockN{Paolo Junior Angeloni}
\IEEEauthorblockA{\textit{Dept. Computer Science \& Software Engineering} \\
\textit{Concordia University}\\
Montreal, Canada \\
p\_ange@live.concordia.ca}
{\footnotesize ID: 25976944}
}

\maketitle

\section{The Triplet-Loss Pipeline}
In this supervised similarity or metric learning, the Triplet-Loss pipeline consists of 
\begin{itemize}
    \item Retrieve images from CUB200\_2011 dataset within \texttt{TripletCUBDataset} class
    \item Transform images into tensors and apply additional augmentations to the training set only
    \item 
    \item 
\end{itemize}
\par\vspace{1em}
...

\section{Training Curves}
...
\begin{itemize}
    \item ...
    \item ...
    \item ...
\end{itemize}

\section{Visualizations}
...
\begin{itemize}
    \item \textbf{Throughput:}
    \item \textbf{BOLA:} 
    \item \textbf{Dynamic mode:}
\end{itemize}

\section{Evaluation Results}
...
\begin{itemize}
    \item ...
    \item ...
    \item ...
    \item ...
    \item ...
\end{itemize}

\begin{table}[H]
\centering
\caption{Training results with ResNet18}
\begin{tabular}{|l|c|}
\hline
\textbf{Metric} & \textbf{Value} \\
\hline
Epoch & ... \\
Loss & ... \\
Cosine Similarity Anchor-Positive & ... \\
Cosine Similarity Anchor-Negative & ... \\
Train Top-1 & ... \\
\hline
\end{tabular}
\label{tab:}
\end{table}

\begin{figure}[H]
    \centering
    % \includegraphics[width=0.48\textwidth]{}
    \caption{...}
    \label{fig:}
\end{figure}

\subsection{Comparison}

...

\textbf{Buffer Level:}  
\par\vspace{1em}
\textbf{Throughput:}  
\par\vspace{1em}
\textbf{Latency:}  

\subsection{Discussion}
...
\par\vspace{1em}
...

\section{Conclusions}
...
\par\vspace{1em}
...
\begin{itemize}
    \item 
    \item 
    \item 
\end{itemize}

\end{document}