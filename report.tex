\documentclass[conference]{IEEEtran}
\IEEEoverridecommandlockouts
% The preceding line is only needed to identify funding in the first footnote. If that is unneeded, please comment it out.
\usepackage{cite}
\usepackage{amsmath,amssymb,amsfonts}
\usepackage{algorithmic}
\usepackage{graphicx}
\usepackage{textcomp}
\usepackage{xcolor}
\usepackage{listings}
\usepackage{url}
\usepackage{float}
\usepackage{hyperref}
\hypersetup{
    colorlinks=true,
    urlcolor=blue,
    linkcolor=blue,
    citecolor=blue
}
\def\BibTeX{{\rm B\kern-.05em{\sc i\kern-.025em b}\kern-.08em
    T\kern-.1667em\lower.7ex\hbox{E}\kern-.125emX}}
\begin{document}

\title{Metric Learning: Triplet-Loss\\
}

\author{\IEEEauthorblockN{Khashayar Zardoui}
\IEEEauthorblockA{\textit{Dept. Computer Science \& Software Engineering} \\
\textit{Concordia University}\\
Montreal, Canada \\
khashayar.zardoui@mail.concordia.ca}
{\footnotesize ID: 40052568}
\and
\IEEEauthorblockN{Paolo Junior Angeloni}
\IEEEauthorblockA{\textit{Dept. Computer Science \& Software Engineering} \\
\textit{Concordia University}\\
Montreal, Canada \\
p\_ange@live.concordia.ca}
{\footnotesize ID: 25976944}
}

\maketitle


\section{The Triplet-Loss Pipeline}
In this supervised similarity or metric learning, the Triplet-Loss pipeline consists of 
\begin{enumerate}
    \item Retrieve images from CUB200\_2011 dataset within \texttt{TripletCUBDataset} class
    \item Transform images into tensors and apply additional augmentations to the training set only
    \item 
    \item 
\end{enumerate}
\par\vspace{1em}


\section{Training Hyper-parameters}
We conducted 4 experiments using two pre-trained models: ResNet18 and ResNet34 \\
ResNet18 contains ... million parameters \\
ResNet34 contains approximately 21.5 million parameters \\
\begin{enumerate}
    \item epochs: 20, learning rate: 0.001, batch size: 32
    \item epochs: 20, learning rate: 0.002, batch size: 32
    \item epochs: 20, learning rate: 0.001, batch size: 64
    \item epochs: 20, learning rate: 0.002, batch size: 64
\end{enumerate}


\section{Training Curves}
\subsection{ResNet18}
\begin{figure}[H]
    \centering
    \includegraphics[width=0.48\textwidth]{loss_graph_34_1.png}
    \caption{Experiment 1 with ResNet18}
    \label{fig:first_resnet18}
\end{figure}

\subsection{ResNet34}
\begin{figure}[H]
    \centering
    \includegraphics[width=0.48\textwidth]{loss_graph_34_1.png}
    \caption{Experiment 1 with ResNet34}
    \label{fig:first_resnet34}
\end{figure}
\begin{figure}[H]
    \centering
    \includegraphics[width=0.48\textwidth]{loss_graph_34_2.png}
    \caption{Experiment 2 with ResNet34}
    \label{fig:second_resnet34}
\end{figure}
\begin{figure}[H]
    \centering
    \includegraphics[width=0.48\textwidth]{loss_graph_34_3.png}
    \caption{Experiment 3 with ResNet34}
    \label{fig:third_resnet34}
\end{figure}


\section{Embedding Visualizations}
\subsection{ResNet18}
\begin{figure}[H]
    \centering
    \includegraphics[width=0.55\textwidth]{train_plot_34_1.png}
    \caption{Experiment 1 with ResNet18}
    \label{fig:first_vis_resnet18}
\end{figure}

\subsection{ResNet34}
\begin{figure}[H]
    \centering
    \includegraphics[width=0.55\textwidth]{train_plot_34_1.png}
    \caption{Experiment 1 with ResNet34}
    \label{fig:first_vis_resnet34}
\end{figure}
\begin{figure}[H]
    \centering
    \includegraphics[width=0.55\textwidth]{train_plot_34_2.png}
    \caption{Experiment 2 with ResNet34}
    \label{fig:second_vis_resnet34}
\end{figure}


\section{Evaluation Results}
\subsection{ResNet18}
\begin{table}[H]
\centering
\caption{Loss and Accuracy Metrics for Experiment 1}
\begin{tabular}{|l|c|}
\hline
\textbf{Metric} & \textbf{Value} \\
\hline
Loss &  \\
Top-1 Accuracy (\%) &  \\
Cosine Similarity Anchor-Positive &  \\
Cosine Similarity Anchor-Negative &  \\
\hline
\end{tabular}
\label{tab:first_stats_resnet18}
\end{table}

\begin{table}[H]
\centering
\caption{Precision Metrics for Experiment 1}
\begin{tabular}{|l|c|}
\hline
\textbf{Metric} & \textbf{Value (\%)} \\
\hline
Precision@1 &  \\
Precision@5 &  \\
Precision@10 &  \\
\hline
\end{tabular}
\label{tab:first_precision_resnet18}
\end{table}

\subsection{ResNet34}

\begin{table}[H]
\centering
\caption{Loss and Accuracy Metrics for Experiment 1}
\begin{tabular}{|l|c|}
\hline
\textbf{Metric} & \textbf{Value} \\
\hline
Loss & 0.5902 \\
Top-1 Accuracy (\%) & 75.85 \\
Cosine Similarity Anchor-Positive & 0.6200 \\
Cosine Similarity Anchor-Negative &  0.0994 \\
\hline
\end{tabular}
\label{tab:first_stats_resnet34}
\end{table}
\begin{table}[H]
\centering
\caption{Precision Metrics for Experiment 1}
\begin{tabular}{|l|c|}
\hline
\textbf{Metric} & \textbf{Value (\%)} \\
\hline
Precision@1 & 31.69 \\
Precision@5 & 30.12 \\
Precision@10 & 30.37 \\
\hline
\end{tabular}
\label{tab:first_precision_resnet34}
\end{table}

\begin{table}[H]
\centering
\caption{Loss and Accuracy Metrics for Experiment 2}
\begin{tabular}{|l|c|}
\hline
\textbf{Metric} & \textbf{Value} \\
\hline
Loss & 0.5829 \\
Top-1 Accuracy (\%) & 75.63 \\
Cosine Similarity Anchor-Positive & 0.6727 \\
Cosine Similarity Anchor-Negative &  0.1282 \\
\hline
\end{tabular}
\label{tab:second_stats_resnet34}
\end{table}
\begin{table}[H]
\centering
\caption{Precision Metrics for Experiment 2}
\begin{tabular}{|l|c|}
\hline
\textbf{Metric} & \textbf{Value (\%)} \\
\hline
Precision@1 & 26.34 \\
Precision@5 & 29.14 \\
Precision@10 & 29.42 \\
\hline
\end{tabular}
\label{tab:second_precision_resnet34}
\end{table}

\begin{table}[H]
\centering
\caption{Loss and Accuracy Metrics for Experiment 3}
\begin{tabular}{|l|c|}
\hline
\textbf{Metric} & \textbf{Value} \\
\hline
Loss & 0.4440 \\
Top-1 Accuracy (\%) & 83.28 \\
Cosine Similarity Anchor-Positive & 0.7201 \\
Cosine Similarity Anchor-Negative & 0.0272 \\
\hline
\end{tabular}
\label{tab:third_stats_resnet34}
\end{table}
\begin{table}[H]
\centering
\caption{Precision Metrics for Experiment 3}
\begin{tabular}{|l|c|}
\hline
\textbf{Metric} & \textbf{Value (\%)} \\
\hline
Precision@1 & 34.98 \\
Precision@5 & 30.86 \\
Precision@10 & 29.55 \\
\hline
\end{tabular}
\label{tab:third_precision_resnet34}
\end{table}

\end{document}